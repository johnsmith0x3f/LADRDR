\chapter{Linear Maps}

\section{The Vector Space of Linear Maps}

\begin{exercise}{10}
	Suppose \( U \) is a subspace of \( V \) with \( U \neq V \). Suppose \( S \in \lmap(U, W) \) and \( S \neq 0 \) (which means that \( Su \neq 0 \) for some \( u \in U \) ). Define \( T \in \lmap(V, W) \) by
	\[
		Tv = \begin{cases}
			Sv & \text{if} \ v \in U, \\
			0  & \text{if} \ v \in V \ \text{and} \ v \not\in U.
		\end{cases}
	\]
Prove that \( T \) is not a linear map on \( V \).
\end{exercise}

\begin{proof}
	Take \( u \in U \) such that \( Su \neq 0 \). Since \( U \neq V \), there exists \( v \in V \setminus U \). Then
	\[
		T(u - v) + Tv = 0 + 0 = 0 \neq Su = Tu.
	\]
	Hence \( T \) is not a linear map.
\end{proof}

\section{Null Spaces and Ranges}

\begin{exercise}{29}
	Suppose \( \varphi \in \lmap(V, \F) \). Suppose \( u \in V \) is not in \( \nullspace{\varphi} \). Prove that
	\[
		V = \nullspace{\varphi} \oplus \{au : a \in \F\}.
	\]
\end{exercise}

\begin{proof}
	For arbitary \( v \in V \), since \( \varphi(u) \neq 0 \), we have
	\[
		\varphi\left(v - \frac{\varphi(v)}{\varphi(u)}u\right) = \varphi(v) - \varphi\left(\frac{\varphi(v)}{\varphi(u)}u\right) = \varphi(v) - \frac{\varphi(v)}{\varphi(u)}\varphi(u) = 0.
	\]
	This implies \( v - \frac{\varphi(v)}{\varphi(u)}u = w \) for some \( w \in \nullspace{\varphi} \), and hence
	\[
		v = w + \frac{\varphi(v)}{\varphi(u)}u \in \nullspace{\varphi} + \{au : a \in \F\}.
	\]
	Therefore, \( V = \nullspace{\varphi} + \{au : a \in \F\} \). Since \( \nullspace{\varphi} \cap \{au : a \in \F\} = \{0\} \), by 1.45, we have \( V = \nullspace{\varphi} \oplus \{au : a \in \F\} \).
\end{proof}

\section{Matrices}

\begin{exercise}{15}
	Suppose \( A \) is an \( n \)-by-\( n \) matrix and \( 1 \le j, k \le n \). Show that the entry in row j, column \( k \), of \( A^{3} \) (which is defined to mean \( AAA \)) is
	\[
		\sum_{p = 1}^{n} \sum_{r = 1}^{n} A_{j,p}A_{p,r}A_{r,k}.
	\]
\end{exercise}

\begin{proof}
	By the definition of matrix multiplication, we have
	\begin{align*}
		(A^{3})_{j,k} &= (AA^{2})_{j,k} \\
									&= \sum_{p = 1}^{n} A_{j,p}(A^{2})_{p,k} \\
									&= \sum_{p = 1}^{n} A_{j,p} \sum_{r = 1}^{n} A_{p,r}A_{r,k} \\
									&= \sum_{p = 1}^{n} \sum_{r = 1}^{n} A_{j,p}A_{p,r}A_{r,k}. \qedhere
	\end{align*}
\end{proof}

\section{Invertibility and Isomorphic Vector Spaces}

\section{Products and Quotients of Vector Spaces}

\section{Duality}
