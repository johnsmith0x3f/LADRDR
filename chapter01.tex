\chapter{Vector Spaces}

%%% R^n and C^n {{{

\section{\texorpdfstring{$\R^{n}$}{R\^{}n} and \texorpdfstring{$\C^{n}$}{C\^{}n}}

\begin{exercise}{7}
	Show that for every \( \alpha \in \C \), there exists a unique \( \beta \in \C \)
	such that \( \alpha + \beta = 0 \).
\end{exercise}

\begin{proof}
	Suppose \( \alpha = a + bi \). Let \( \beta = -a - bi \). Then
	\[
		\alpha + \beta = (a - a) + (b - b)i = 0,
	\]
	proving existence.

	Suppose there exists \( \gamma \in \C \) such that \( \alpha + \gamma = 0 \). Then
	\[
		\gamma = \gamma + (\alpha + \beta) = (\gamma + \alpha) + \beta = \beta,
	\]
	proving uniqueness.
\end{proof}

%%% }}}

%%% Definition of Vector Space {{{

\section{Definition of Vector Space}

\begin{exercise}{6}
	Let \( \infty \) and \( -\infty \) denote two distinct objects, neither of which is in \( \R \). Define an addition and scalar multiplication on \( \R \cup \{\infty\} \cup \{-\infty\} \) as you could guess from the notation. Specifically, the sum and product of two real numbers is as usual, and for
	\( t \in \R \) define
	\begin{gather*}
		t\infty = \begin{cases}
			-\infty & \text{if} \ t < 0, \\
			0       & \text{if} \ t = 0, \\
			\infty  & \text{if} \ t > 0,
		\end{cases} \quad
		t(-\infty) = \begin{cases}
			\infty  & \text{if} \ t < 0, \\
			0       & \text{if} \ t = 0, \\
			-\infty & \text{if} \ t > 0,
		\end{cases} \\
		t + \infty = \infty + t = \infty, \quad\quad\quad t + (-\infty) = (-\infty) + t = -\infty, \\
		\infty + \infty = \infty, \quad\quad (-\infty) + (-\infty) = -\infty, \quad\quad \infty + (-\infty) = 0.
	\end{gather*}
	Is \( \R \cup \{\infty\} \cup \{-\infty\} \) a vector space over \( \R \)? Explain.
\end{exercise}

\begin{solution}
	No. If \( \R \cup \{\infty\} \cup \{-\infty\} \) is a vector space over \( \R \), we will have
	\[
		1 = 1 + 0 = 1 + (\infty + (-\infty)) = (1 + \infty) + (-\infty) = \infty + (-\infty) = 0,
	\]
	a contradiction.
\end{solution}

%%% }}}

%%% Subspaces {{{

\section{Subspaces}

\begin{exercise}{9}
A function \( f: \R \to \R \) is called \textbf{\textit{periodic}} if there exists a positive number \( p \) such that \( f(x) = f(x + p) \) for all \( x \in \R \). Is the set of periodic functions from \( \R \) to \( \R \) a subspace of \( \R^{\R} \)? Explain.
\end{exercise}

\begin{solution}
	No. Let \( F_{p} \) denote the set of periodic functions from \( \R \) to \( \R \). If \( F_{p} \) is a subspace of \( \R^{\R} \), then \( h(x) = \cos x + \sin\pi x \in F_{p} \) since both \( f(x) = \cos x \) and \( g(x) = \sin\pi x \) are in \( F_{p} \). In other words, there exists \( p > 0 \) such that
	\[
		\cos p - \sin\pi p = 1 = \cos p + \sin\pi p.
	\]
	Hence we have \( \cos p = 1 \) and \( \sin\pi p = 0 \). The former implies \( p = 2n\pi (n \in \Z_{+}) \), while the latter implies \( p = m (m \in \Z_{+}) \). However, this means
	\[
		\pi = \frac{m}{2n} \in \Q,
	\]
	which is impossible.
\end{solution}

\begin{exercise}{12}
	Prove that the union of two subspaces of \( V \) is a subspace of \( V \) if and only if one of the subspaces is contained in the other.
\end{exercise}

\begin{proof}
	Suppose \( U \) and \( W \) are two subspaces of \( V \), such that \( U \cup W \) is also a subspace and neither is contained in the other. In other words, \( U \not\subset W \) and \( W \not\subset U \). Let \( u \in U - W \) and \( w \in W - U \). Since \( U \cup W \) is a subspace of \( V \), we have \( u + w \in U \cup W \). However,
	\begin{align*}
		u + w \in U \quad &\Rightarrow \quad w = (u + w) - u \in U, \\
		u + w \in W \quad &\Rightarrow \quad u = (u + w) - w \in W,
	\end{align*}
	which implies that \( u + w \) is in neither \( U \) nor \( W \), a contradiction.

	Conversely, suppose one of \( U \) and \( W \) is contained in the other. Without loss of generality, we assume \( U \subset W \). Then \( U \cup W = W \) is obviously a subspace of \( V \).
\end{proof}

\begin{exercise}{13}
	Prove that the union of three subspaces of \( V \) is a subspace of \( V \) if and only if one of the subspaces contains the other two.

	\note{This exercise is surprisingly harder than the previous exercise, possibly because this exercise is not true if we replace \( \F \) with a field containing only two elements.}
\end{exercise}

\begin{proof}
	One direction is obvious. Without loss of generality, suppose \( A, B, C \) are three subspaces of \( V \) satisfying \( A \cup B \subset C \). Then \( A \cup B \cup C = C \) is a subspace of \( V \).
	
	Suppose \( A \not\subset B \cup C \). Let \( v \in A - (B \cup C) \) and \( w \in B - A \). Then \( v + w \) is in neither \( A \) nor \( B \), and hence is in \( C \). Consider the vector \( 2v \). For any subspace \( S \), \( 2v \in S \) if and only if \( v \in S \). Hence \( 2v \in A - (B \cup C) \), which implies \( 2v + w \in C \). However, this deduces
	\[
		v = (2v + w) - (v + w) \in C,
	\]
	a contradiction.

%	To prove the other direction, one may notice that \( A \cup B \cup C = A \cup (B \cup C) \), and hence may try using the same approach as in the previous exercise, where one would get stuck for \( B \cup C \) is not neccessarily a subspace of \( V \). However, Exercise 11 informs that, fortunately, \( B \cap C \) is a subspace of \( V \).

%	Let \( u \in A - (B \cup C), a = (B \cap C) - A \). If \( u + a \in A \), then we have \( a \in A \), a contradiction. If \( u + a \in B \cap C \), then we have \( u \in B \cap C \subset B \cup C \), also a contradiction. Hence \( u + a \in S - (A \cup (B \cap C)) \)
\end{proof}

%%% }}}
