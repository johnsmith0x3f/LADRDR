%! TEX TS-program = xelatex
%! TEX encoding = UTF-8

\documentclass[oneside]{book}

\usepackage{matheo}

\DeclareMathOperator{\dd}{d}

\DeclareMathAlphabet{\mathcal}{OMS}{zplm}{m}{n}
\DeclareMathOperator{\Lmap}{\mathcal{L}}
\DeclareMathOperator{\Poly}{\mathcal{P}}

\newcommand{\conj}{\overline}

\newcommand{\note}[1]{[\textit{#1}]}

\DeclareMathOperator{\nul}{null}

% Set the fonts.
\usepackage{fontspec}
\setmainfont{Times New Roman}
\setsansfont{Arial}
\usepackage{newtxtext}
\usepackage{newtxmath}

\usepackage{titlesec}
\renewcommand{\thesection}{\arabic{chapter}.\Alph{section}}

\usepackage[many]{tcolorbox}
\definecolor{exercisemain}{HTML}{CA6C37}
\definecolor{exerciseback}{HTML}{FEF9DA}
\newtcolorbox{exercise}[1]{
	title = {Exercise #1},
	fonttitle = {\sffamily\bfseries},
	coltitle = exercisemain,
	attach title to upper = {\quad}, % title
	colback = exerciseback, % background
	sharp corners,
	enhanced jigsaw,
	boxrule = 0pt,
	toprule = 2pt,
	bottomrule = 2pt,
	colframe = exercisemain % frame
}

\usepackage{hyperref}

\title{LADR Done Right}
\author{johnsmith0x3f}
\date{\today}

\AtBeginDocument{
  \mathchardef\stdcomma=\mathcode`,
  \mathcode`,="8000
}
\begingroup\lccode`~=`, \lowercase{\endgroup\def~}{\stdcomma\,}

\begin{document}

\maketitle
\tableofcontents

\chapter{Vector Spaces}

%%% R^n and C^n {{{

\section{\texorpdfstring{$\R^{n}$}{R\^{}n} and \texorpdfstring{$\C^{n}$}{C\^{}n}}

\begin{exercise}{7}
	Show that for every \( \alpha \in \C \), there exists a unique \( \beta \in \C \)
	such that \( \alpha + \beta = 0 \).
\end{exercise}

\begin{proof}
	Suppose \( \alpha = a + bi \). Let \( \beta = -a - bi \). Then
	\[
		\alpha + \beta = (a - a) + (b - b)i = 0,
	\]
	proving existence.

	Suppose there exists \( \gamma \in \C \) such that \( \alpha + \gamma = 0 \). Then
	\[
		\gamma = \gamma + (\alpha + \beta) = (\gamma + \alpha) + \beta = \beta,
	\]
	proving uniqueness.
\end{proof}

%%% }}}

%%% Definition of Vector Space {{{

\section{Definition of Vector Space}

\begin{exercise}{6}
	Let \( \infty \) and \( -\infty \) denote two distinct objects, neither of which is in \( \R \). Define an addition and scalar multiplication on \( \R \cup \{\infty\} \cup \{-\infty\} \) as you could guess from the notation. Specifically, the sum and product of two real numbers is as usual, and for
	\( t \in \R \) define
	\begin{gather*}
		t\infty = \begin{cases}
			-\infty & \text{if} \ t < 0, \\
			0       & \text{if} \ t = 0, \\
			\infty  & \text{if} \ t > 0,
		\end{cases} \quad
		t(-\infty) = \begin{cases}
			\infty  & \text{if} \ t < 0, \\
			0       & \text{if} \ t = 0, \\
			-\infty & \text{if} \ t > 0,
		\end{cases} \\
		t + \infty = \infty + t = \infty, \quad\quad\quad t + (-\infty) = (-\infty) + t = -\infty, \\
		\infty + \infty = \infty, \quad\quad (-\infty) + (-\infty) = -\infty, \quad\quad \infty + (-\infty) = 0.
	\end{gather*}
	Is \( \R \cup \{\infty\} \cup \{-\infty\} \) a vector space over \( \R \)? Explain.
\end{exercise}

\begin{solution}
	No. If \( \R \cup \{\infty\} \cup \{-\infty\} \) is a vector space over \( \R \), we will have
	\[
		1 = 1 + 0 = 1 + (\infty + (-\infty)) = (1 + \infty) + (-\infty) = \infty + (-\infty) = 0,
	\]
	a contradiction.
\end{solution}

%%% }}}

%%% Subspaces {{{

\section{Subspaces}

\begin{exercise}{9}
A function \( f: \R \to \R \) is called \textbf{\textit{periodic}} if there exists a positive number \( p \) such that \( f(x) = f(x + p) \) for all \( x \in \R \). Is the set of periodic functions from \( \R \) to \( \R \) a subspace of \( \R^{\R} \)? Explain.
\end{exercise}

\begin{solution}
	No. Let \( F_{p} \) denote the set of periodic functions from \( \R \) to \( \R \). If \( F_{p} \) is a subspace of \( \R^{\R} \), then \( h(x) = \cos x + \sin\pi x \in F_{p} \) since both \( f(x) = \cos x \) and \( g(x) = \sin\pi x \) are in \( F_{p} \). In other words, there exists \( p > 0 \) such that
	\[
		\cos p - \sin\pi p = 1 = \cos p + \sin\pi p.
	\]
	Hence we have \( \cos p = 1 \) and \( \sin\pi p = 0 \). The former implies \( p = 2n\pi (n \in \Z_{+}) \), while the latter implies \( p = m (m \in \Z_{+}) \). However, this means
	\[
		\pi = \frac{m}{2n} \in \Q,
	\]
	which is impossible.
\end{solution}

\begin{exercise}{12}
	Prove that the union of two subspaces of \( V \) is a subspace of \( V \) if and only if one of the subspaces is contained in the other.
\end{exercise}

\begin{proof}
	Suppose \( U \) and \( W \) are two subspaces of \( V \) such that \( U \cup W \) is also a subspace of \( V \). If \( U \subseteq W \), then there is nothing to proof. Otherwise, take \( u \in U \setminus W, w \in W \) and consider \( u + w \in U \cup W \). It cannot be in \( W \), since then will be \( u = (u + w) - w \in W \). Therefore, there must be \( u + w \in U \), and hence \( w = (u + w) - u \in U \), which implies \( W \subseteq U \), as desired.

	Conversely, suppose one of \( U \) and \( W \) is contained in the other. Without loss of generality, we assume \( U \subseteq W \). Then \( U \cup W = W \) is obviously a subspace of \( V \).
\end{proof}

\begin{exercise}{13}
	Prove that the union of three subspaces of \( V \) is a subspace of \( V \) if and only if one of the subspaces contains the other two.

	\note{This exercise is surprisingly harder than the previous exercise, possibly because this exercise is not true if we replace \( \F \) with a field containing only two elements.}
\end{exercise}

\begin{proof}
	Suppose \( A, B, C \) are three subspaces of \( V \) such that \( S = A \cup B \cup C \) is also a subspace of \( V \).

	If \( A \subseteq B \), then by Exercise 12, one of \( B \subseteq C \)
	
	Let \( v \in A - (B \cup C) \) and \( w \in B - A \). Then \( v + w \) is in neither \( A \) nor \( B \), and hence is in \( C \). Consider the vector \( 2v \). For any subspace \( S \), \( 2v \in S \) if and only if \( v \in S \). Hence \( 2v \in A - (B \cup C) \), which implies \( 2v + w \in C \). However, this deduces
	\[
		v = (2v + w) - (v + w) \in C,
	\]
	a contradiction.

	One direction is obvious. Without loss of generality, suppose \( A, B, C \) are three subspaces of \( V \) satisfying \( A \cup B \subseteq C \). Then \( A \cup B \cup C = C \) is a subspace of \( V \).
\end{proof}

%%% }}}


\chapter{Finite-Dimensional Vector Spaces}

\section{Span and Linear Independence}

\begin{exercise}{10}
	Suppose \( v_{1}, \ldots, v_{m} \) is linearly independent in \( V \) and \( w \in V \). Prove that if \( v_{1} + w, \ldots, v_{m} + w \) is linearly dependent, then \( w \in \spanspace(v_{1}, \ldots, v_{m}) \).
\end{exercise}

\begin{proof}
	If \( v_{1} + w, \ldots, v_{m} + w \) is linearly dependent, then there exists \( a_{1}, \ldots, a_{m} \), not all \( 0 \), such that
	\begin{align*}
		a_{1}(v_{1} + w) + \cdots + a_{m}(v_{m} + w) &= 0, \\
		            a_{1}v_{1} + \cdots + a_{m}v_{m} &= -(a_{1} + \cdots + a_{m})w.
	\end{align*}
	Since \( v_{1} + w, \ldots, v_{m} + w \) is linearly dependent, \( a_{1} + \cdots + a_{m} \neq 0 \), and hence
	\[
		w = \frac{-a_{1}}{a_{1} + \cdots + a_{m}}v_{1} + \cdots + \frac{-a_{m}}{a_{1} + \cdots + a_{m}}v_{m} \in \spanspace(v_{1}, \ldots, v_{m}),
	\]
	as desired.
\end{proof}

\section{Bases}

\begin{exercise}{8}
	Suppose \( U \) and \( W \) are subspaces of \( V \) such that \( V = U \oplus W \). Suppose also that \( u_{1}, \ldots, u_{m} \) is a basis of \( U \) and \( w_{1}, \ldots, w_{n} \) is a basis of \( W \). Prove that
	\[
		u_{1}, \ldots, u_{m}, w_{1}, \ldots, w_{n}
	\]
	is a basis of \( V \).
\end{exercise}

\begin{proof}
	Consider an arbitary \( v \in V \). Since \( V = U \oplus W \), there are unique vectors \( u \in U \) and \( w \in W \) such that \( v = u + w \). Hence there are unique scalars \( a_{1}, \ldots, a_{m}, b_{1}, \ldots, b_{n} \) such that
	\[
		v = u + w = a_{1}u_{1} + \cdots + a_{m}u_{m} + b_{1}w_{1} + \cdots + b_{n}w_{n},
	\]
	which, by 2.29, implies that \( u_{1}, \ldots, u_{m}, w_{1}, \ldots, w_{n} \) is a basis of \( V \).
\end{proof}

\section{Dimension}

\begin{exercise}{17}
	You might guess, by analogy with the formula for the number of elements in the union of three subsets of a finite set, that if \( U_{1}, U_{2}, U_{3} \) are subspaces of a finite-dimensional vector space, then
	\begin{align*}
		\dim(U_{1} + U_{2} + U_{3}) =& \dim{U_{1}} + \dim{U_{2}} + \dim{U_{3}} \\
		                             &- \dim(U_{1} \cap U_{2}) - \dim(U_{2} \cap U_{3}) - \dim(U_{3} \cap U_{1}) \\
																 &+ \dim(U_{1} \cap U_{2} \cap U_{3}).
	\end{align*}
	Prove this or give a counterexample.
\end{exercise}

\begin{countexam}
	Consider the \( \R^{2} \) plane. Let \( V_{1}, V_{2}, V_{3} \) be three distinct lines through the origin. Then \( \LHS = \dim{\R^{2}} = 2 \), while \( \RHS = 1 + 1 + 1 - 0 - 0 - 0 + 0 = 3 \).
\end{countexam}


\chapter{Inner Product Spaces}

\section{Inner Products and Norms}

\begin{exercise}{5}
	Suppose $T \in \Lmap(V)$ is such that $\norm{Tv} \le \norm{v}$ for every $v \in V$. Prove that $T-\sqrt{2}I$ is invertible.
\end{exercise}

\begin{proof}
	Suppose $V$ is finite-dimensional. Suppose $u \in \nul(T - \sqrt{2}I)$. Then we have $Tu = \sqrt{2}u$, and hence $\norm{Tu} = \norm{\sqrt{2}u} = \sqrt{2}\norm{u}$. Since $\norm{tv} \le \norm{v}$, this implies that $u = 0$.
\end{proof}

\begin{exercise}{6}
	Suppose $u, v \in \Lmap(V)$. Prove that $\inp{u}{v} = 0$ if and only if
	\[
		\norm{u} \le \norm{u + av}
	\]
	for all $a \in \F$.
\end{exercise}

\begin{solution}
	
\end{solution}

\begin{exercise}{8}
	Suppose $u, v \in V$ and $\norm{u} = \norm{v} = 1$ and $\inp{u}{v} = 1$. Prove that $u = v$.
\end{exercise}

\begin{proof}
	Note that
	\[
		\norm{u - v}^{2} = \inp{u - v}{u - v}
		                 = \norm{u}^{2} - \inp{u}{v} - \inp{v}{u} + \norm{v}^{2}
		                 = 0.
	\]
	Hence $u - v = 0$ by definiteness.
\end{proof}



\end{document}
