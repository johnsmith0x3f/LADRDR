\chapter{Inner Product Spaces}

\section{Inner Products and Norms}

\begin{exercise}{5}
	Suppose \( T \in \Lmap(V) \) is such that \( \norm{Tv} \le \norm{v} \) for every \( v \in V \). Prove that \( T-\sqrt{2}I \) is invertible.
\end{exercise}

\begin{proof}
	Suppose \( V \) is finite-dimensional. For all \( u \in \nullspace(T - \sqrt{2}I) \), we have \( Tu = \sqrt{2}u \), and hence \( \norm{Tu} = \norm{\sqrt{2}u} = \sqrt{2}\norm{u} \). Since \( \norm{tv} \le \norm{v} \), this implies that \( u = 0 \).
\end{proof}

\begin{exercise}{6}
	Suppose \( u, v \in \Lmap(V) \). Prove that \( \inp{u}{v} = 0 \) if and only if
	\[
		\norm{u} \le \norm{u + av}
	\]
	for all \( a \in \F \).
\end{exercise}

\begin{solution}
	If \( \inp{u}{v} = 0 \), then
	\[
		\norm{u + av}^{2} - \norm{u}^{2} = \norm{av}^{2} \ge 0.
	\]
	by the Pythagorean Theorem.

	If \( \norm{u} \le \norm{u + av} \) for all \( a \in \F \), then
	\[
		0 \le \norm{u + av}^{2} - \norm{u}^{2} = \conj{a}\inp{u}{v} + a\conj{\inp{u}{v}} + \abs{a}^{2}\norm{v}^{2}.
	\]
	Letting \( a = -\frac{\inp{u}{v}}{\norm{v}^{2}} \) yields \( \inp{u}{v} = 0 \).
\end{solution}

\begin{exercise}{8}
	Suppose \( u, v \in V \) and \( \norm{u} = \norm{v} = 1 \) and \( \inp{u}{v} = 1 \). Prove that \( u = v \).
\end{exercise}

\begin{proof}
	Note that
	\[
		\norm{u - v}^{2} = \inp{u - v}{u - v} = \norm{u}^{2} - \inp{u}{v} - \inp{v}{u} + \norm{v}^{2} = 0.
	\]
	Hence \( u - v = 0 \) by definiteness.
\end{proof}

\begin{exercise}{11}
	Prove that
	\[
		16 \le (a + b + c + d)\left(\frac{1}{a} + \frac{1}{b} + \frac{1}{c} + \frac{1}{d}\right)
	\]
	for all positive numbers \( a, b, c, d \).
\end{exercise}

\begin{proof}
	By Exercise 6.17(a), we have
	\begin{align*}
		16 &= \abs{\sqrt{a \cdot \frac{1}{a}} + \sqrt{b \cdot \frac{1}{b}} + \sqrt{c \cdot \frac{1}{c}} + \sqrt{d \cdot \frac{1}{d}}}^{2} \\
			 &\le (a + b + c + d)\left(\frac{1}{a} + \frac{1}{b} + \frac{1}{c} + \frac{1}{d}\right). \qedhere
	\end{align*}
\end{proof}

\begin{exercise}{12} Prove that \[
		(x_{1} + x_{2} + \cdots + x_{n})^{2} \le n\left(x_{1}^{2} + x_{2}^{2} + \cdots + x_{n}^{2}\right)
	\]
	for all positive integers \( n \) and all real numbers \( x_{1}, \ldots, x_{n} \).
\end{exercise}

\begin{proof}
	In Exercise 6.17(a), let \( y_{1} = y_{2} = \cdots = y_{n} = 1 \).
\end{proof}

\begin{exercise}{17}
	Prove or disprove: there is an inner product on \( \R^{2} \) such that the associated norm is given by
	\[
		\norm{(x, y)} = \max\{x, y\}
	\]
	for all \( (x, y) \in \R^{2} \).
\end{exercise}

\begin{countexam}
	Let \( u = (0, 1), v = (1, 0) \). Then 6.22 fails.
\end{countexam}

\section{Orthonormal Bases}

\begin{exercise}
	Find a polynomial \( q \in \Poly_{2}(\R) \) such that
	\[
		p\left(\frac{1}{2}\right) = \int_{0}^{1} p(x)q(x)\dd{x}
	\]
	for every \( p \in \Poly_{2}(\R) \).
\end{exercise}

\begin{solution}
	Let \( \varphi(p) = p\left(\frac{1}{2}\right) \) and \( \inp{p}{q} = \int_{0}^{1} p(x)q(x)\dd{x} \). Then with the orthonormal basis found in Exercise 5 and the formula in 6.43, we can find that
	\[
		q(x) = -15x^{2} + 15x - \frac{3}{2}. \qedhere
	\]
\end{solution}
