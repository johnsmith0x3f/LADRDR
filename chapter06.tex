\chapter{Inner Product Spaces}

\section{Inner Products and Norms}

\begin{exercise}{5}
	Suppose \( T \in \Lmap(V) \) is such that \( \norm{Tv} \le \norm{v} \) for every \( v \in V \). Prove that \( T-\sqrt{2}I \) is invertible.
\end{exercise}

\begin{proof}
	Suppose \( V \) is finite-dimensional. For all \( u \in \nul(T - \sqrt{2}I) \), we have \( Tu = \sqrt{2}u \), and hence \( \norm{Tu} = \norm{\sqrt{2}u} = \sqrt{2}\norm{u} \). Since \( \norm{tv} \le \norm{v} \), this implies that \( u = 0 \).
\end{proof}

\begin{exercise}{6}
	Suppose \( u, v \in \Lmap(V) \). Prove that \( \inp{u}{v} = 0 \) if and only if
	\[
		\norm{u} \le \norm{u + av}
	\]
	for all \( a \in \F \).
\end{exercise}

\begin{solution}
	If \( \inp{u}{v} = 0 \), then
	\[
		\norm{u + av}^{2} - \norm{u}^{2} = \norm{av}^{2} \ge 0.
	\]
	by the Pythagorean Theorem.

	If \( \norm{u} \le \norm{u + av} \) for all \( a \in \F \), then
	\[
		0 \le \norm{u + av}^{2} - \norm{u}^{2} = \conj{a}\inp{u}{v} + a\conj{\inp{u}{v}} + \abs{a}^{2}\norm{v}^{2}.
	\]
	Letting \( a = -\frac{\inp{u}{v}}{\norm{v}^{2}} \) yields \( \inp{u}{v} = 0 \).
\end{solution}

\begin{exercise}{8}
	Suppose \( u, v \in V \) and \( \norm{u} = \norm{v} = 1 \) and \( \inp{u}{v} = 1 \). Prove that \( u = v \).
\end{exercise}

\begin{proof}
	Note that
	\[
		\norm{u - v}^{2} = \inp{u - v}{u - v} = \norm{u}^{2} - \inp{u}{v} - \inp{v}{u} + \norm{v}^{2} = 0.
	\]
	Hence \( u - v = 0 \) by definiteness.
\end{proof}

\begin{exercise}{11}
	Prove that
	\[
		16 \le (a + b + c + d)\left(\frac{1}{a} + \frac{1}{b} + \frac{1}{c} + \frac{1}{d}\right)
	\]
	for all positive numbers \( a, b, c, d \).
\end{exercise}

\begin{proof}
	Let \( u = \left(\sqrt{a}, \sqrt{b}, \sqrt{c}, \sqrt{d}\right), v = \left(\frac{1}{\sqrt{a}}, \frac{1}{\sqrt{b}}, \frac{1}{\sqrt{c}}, \frac{1}{\sqrt{d}}\right) \). Then it follows directly from the Cauchy-Schuarz Inequality.
\end{proof}

\begin{exercise}{17}
	Prove or disprove: there is an inner product on \( \R^{2} \) such that the associated norm is given by
	\[
		\norm{(x, y)} = \max\{x, y\}
	\]
	for all \( (x, y) \in \R^{2} \).
\end{exercise}

\begin{proof}[Counterexample]
	Let \( u = (0, 1), v = (1, 0) \). Then 6.22 fails.
\end{proof}
